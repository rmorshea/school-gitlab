\documentclass[prb,preprint]{revtex4-1} 
\raggedbottom
\usepackage{amsmath}
\usepackage{amsfonts}
\usepackage{graphicx}

\begin{document}

\section{Experimental Design (a Summary)}

The experiment which was conducted sought to measure the speed of light ($c$) in three different ways.  These methods measures $c$ through through various mediums and to different ends. The first method implements a simple arrangement of a pulse generator, laser, mirror, and detector to analyze the phase shift of the signal and detected pulses through an oscilloscope effectively measuring $c$ through air. The second method uses an identical arrangement with the exception of the mirror and instead implements a fiber optic cable to direct the light to the detector thus measuring $c$ through the fiber material. The third and final method applies the pulse generator and oscilloscope in a similar fashion to a coaxial cable and variable resistor. By minimizing reflections seen through the cable through impedance matching with a variable resistor one can determine the impedance of the wire itself. Then by measuring the phase shift of the signal and reflection, the reduction in $c$ due to the medium can be determined by comparing it to the propagation of light through a vacuum.

\section{Results And Analysis for the Speed of Light}

For our measurement of $c$ in air we find the value to be $3.0\pm0.3\times10^8$ m/s which is in agreement with the known value of $2.997\times10^8$ m/s. We derived this value by accounting for a measured instrumental lag time of 305$\pm2$ ns to produce a total travel time for the light of 33$\pm3$ ns. Then given that the distance traveled was 10.0� 0.2 m determining the speed is quite simple. For $c$ through the fiber optic cable we find the value to be $2.4\pm0.3\times10^8$ m/s given a traversal time and distance of 12.0$\pm$0.8 ns and 2.127$\pm$0.001 m respectively.  This value then implies that the index of refraction for the the optical fiber is $1.67\pm0.09$ which falls in line with the expected, but approximate value of 1.5. When measuring $c$ through the coaxial cable, it was found that the characteristic impedance of the cable was determined to be 100$\pm2$ ohms, where reflections are apparent for resistances which are both higher and lower than the characteristic impedance of the cable. The value of $c$ through the coaxial cable was then determined to be $2.34\pm0.04\times10^8$ m/s which is equivalently $(0.78\pm0.01)c$ using a measured travel time of 2.600$\pm0.004$ $\mu$s
and a known cable length of 1000 ft or 304.8 m.

\end{document}