\documentclass[prb,preprint]{revtex4-1} 
\raggedbottom
\usepackage{amsmath}
\usepackage{amsfonts}
\usepackage{graphicx}

\begin{document}

\title{Studying The Nature of Radiation}

\author{Ryan S. Morshead}

\affiliation{Department of Physics, California State Polytechnic University}

\date{\today}

\begin{abstract}




\end{abstract}


\maketitle


\section{Introduction}

Radiation physics and chemistry flourished at the end of the 20th century; Wilhelm Conrad Rontgen made his sensational discovery of X rays in 1895 and it was soon followed by Henri Becquerel's discovery of radioactivity and by J. J. Thomson?s proof of the independent existence of negative electrons of small mass. Then Marie and Pierre Curie discovered the radioactive elements polonium and radium, and Paul Villard made the first detailed observations of gamma rays. Though Paul Villard is almost forgotten today by the scientific community. He is given credit for having discovered gamma rays, but that discovery is almost never discussed in any detail. He stands, one might say, in the shadow of giants.

It was on May 18$^{th}$ that Villard demonstrated that radium emits rays which are non-deviable under the influence of magnetic fields and are capable of penetrating dense objects. These new rays, said Villard, were different from the radium rays observed so far. He went on to suggest that the extremely penetrating rays, were similar to the X rays previously observed by Rontgen and thus he claimed they were a kind of X ray.

This paper uses the penetrating capabilities of gamma rays which were first observed by Villard to determine the mass attenuation coefficient for lead and aluminum.



\section{Experimental Design}



\section{Conclusion}

Consectetur adipiscing elit. Donec placerat velit augue, in ultrices lacus pharetra sit amet. Nullam blandit blandit nibh eu feugiat. Curabitur luctus purus vel nunc interdum, in congue eros rutrum. Fusce in urna a massa convallis mollis vel eget nulla. Nam porttitor ipsum a ante luctus bibendum. Nam id lacus quis est placerat porttitor ut ut risus. Donec rhoncus cursus justo in tempus. Pellentesque habitant morbi tristique senectus et netus et malesuada fames ac turpis egestas. Curabitur tristique sagittis augue, in vestibulum metus bibendum vitae. Quisque id condimentum ipsum. Donec porttitor sem eget nisi iaculis, a convallis arcu tristique. Etiam dignissim commodo velit et consectetur.

\begin{acknowledgments}

Proin malesuada sem nec ipsum elementum pellentesque at a arcu. Etiam sollicitudin, purus in mollis placerat, lacus risus facilisis nisl, quis tempor dolor lectus ac leo.

\end{acknowledgments}


\begin{thebibliography}{99}

\bibitem{feynman} Richard P. Feynman, Robert B. Leighton, and Matthew Sands, 
\textit{The Feynman Lectures on Physics, Vol.\ 1} (Addison-Wesley, 1964), p.~3-10.

\bibitem{thermoacoustics} J. Wheatley, T. Hofler, G. W. Swift, and A. Migliori, ``Understanding
some simple phenomena in thermoacoustics with applications to acoustical
heat engines? Am. J. Phys. 53, 147--162 (1985).

\bibitem{dyson} Freeman J. Dyson, ``Feynman's proof of the Maxwell equations,''
Am. J. Phys. \textbf{58} (3), 209--211.  

\bibitem{AIPstylemanual} \textit{AIP Style Manual}, 4th edition (American 
Institute of Physics, New York, 1990). Available online at 
\url{<http://www.aip.org/pubservs/style/4thed/toc.html>}. Although parts of 
it have been made out of date by advancing technology, most of this manual 
is still as useful as ever. Just be sure to follow AJP's specific rules
whenever they conflict with those in the manual.

\end{thebibliography}

\end{document}
